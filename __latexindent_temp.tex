\documentclass[a4paper]{article}

\usepackage[spanish]{babel}
\usepackage{listings}
\usepackage[utf8]{inputenc}
\usepackage{titling}
\usepackage{enumitem}
\usepackage{fancyhdr}
\usepackage{xcolor}
\usepackage{geometry}
\usepackage{graphicx}
\usepackage{hyperref}
\usepackage{cite}
\usepackage{url}

%%search -> (?:url=\{)(.*)(:?\})
%%replace -> howpublished={\\url{$1}}
\geometry{a4paper, margin=7em}


\lstset{
    frame=single,
    breaklines=true,
    numbers=left,
    keywordstyle=\color{blue},
    numbersep=15pt,
    numberstyle=,
    basicstyle=\linespread{1.5}\selectfont\ttfamily,
    commentstyle=\color{gray},
    stringstyle=\color{orange},
    identifierstyle=\color{green!40!black},
}

\setlength{\parindent}{4em}
%%\setlength{\parindent}{0em}
\setlength{\parskip}{0.8em}
    
%%\renewcommand{\familydefault}{phv} %%Seleccionamos Helvetica
    
\lstdefinestyle{console}
{
    numbers=left,
    backgroundcolor=\color{violet},
    %%belowcaptionskip=1\baselineskip,
    breaklines=true,
    %%xleftmargin=\parindent,
    %%showstringspaces=false,
    basicstyle=\footnotesize\ttfamily,
    %%keywordstyle=\bfseries\color{green!40!black},
    %%commentstyle=\itshape\color{green},
    %%identifierstyle=\color{blue},
    %%stringstyle=\color{orange},
    basicstyle=\scriptsize\color{white}\ttfamily,
}
    
\title{Interactiva I}
\author{Aldán Creo Mariño}
    
    
\pagestyle{fancy}
\fancyfoot[R]{\thepage}
\fancyfoot[C]{}
\makeatletter
\let\runauthor\@author
\let\runtitle\@title
\makeatother
\fancyhead[L]{\runauthor}
\fancyhead[C]{OXE}
\fancyhead[R]{Interactiva 1}

\bibliographystyle{acm}
    
\begin{document}
\maketitle

\section{Actitudes emprendedoras}
\subsection{Cales son as principais características a considerar para avaliar o perfil emprendedor?}

Las principales características del perfil emprendedor, de acuerdo con \cite{RefWorks:doc:602fadc2c9e77c0001379bac}, son:

\begin{itemize}
    \item Adaptabilidad
    \item Autonomía
    \item Capacidad de asumir riesgos
    \item Confianza en sí mismo
    \item Fijación continua de objetivos
    \item Innovación
    \item Locus de control interno
    \item Perseverancia Poder de persuasión
    \item Proactividad
    \item Tolerancia a la incertidumbre
\end{itemize}

\subsection{Elabora un listado das motivacións para emprender. Busca un estudio que as clasifique por orde de importancia.}

Existen dos tipos de fuerzas que mueven al emprendedor, según se indica en \cite{RefWorks:doc:602fa427c9e77c0001379ab3}. Por un lado, existen los factores `pull': independencia, realización personal, riqueza, y otros resultados deseables. Por otro, hay factores `push': fuerzas negativas externas que mueven al emprendedor: descontento en su trabajo, problemas para encontrar empleo, salario insuficiente, o horarios no satisfactorios.

De acuerdo con \cite{RefWorks:doc:602fab10c9e77c0001379b54}, los factores `pull' parecen ser más relevantes a la hora de motivar a los emprendedores.

\subsection{Elabora un listado dos riscos para emprender. Busca un estudio que as clasifique por orde de importancia.}

\section{Busca 2 exemplos para innovacións de produto, de proceso, de modelo de negocio e resúmeos brevemente (2 de cada tipo de innovación)}

Innovaciones de producto:
\begin{itemize}
    \item Microondas. Se basó en una tecnología novedosa para llevar a los hogares un invento que mejoraba considerablemente su calidad de vida.
    \item Notas adhesivas. Juntaron dos elementos ya descubiertos (el papel y un tipo de adhesivo con poco poder de adherencia), para crear un nuevo producto cuyo valor es precisamente el hecho de que no es muy adherente, con lo que las notas se pueden despegar con facilidad.
\end{itemize}

Innovaciones de proceso:
\begin{itemize}
    \item Desarrollo de la cadena de montaje. La primera en hacerlo fue Ford, y fue rápidamente copiada por otras empresas, ya que su proceso de producción era capaz de aumentar muy considerablemente el rendimiento de sus fábricas.
    \item 
\end{itemize}

\section{Define ``Análise PEST-PESTEL'' e busca un exemplo onde se aplique.}

El análisis PEST-PESTEL analiza el entorno para, en base a él, tratar de marcar las posibilidades de éxito. Concretamente, analiza variables políticas, económicas, sociales, tecnológicas, y adicionalmente puede analizar variables ambientales (`environmental') y legales.

Un ejemplo se encuentra en \cite{RefWorks:doc:602fb2058f0825dac8eafc64}. Allí, se habla sobre el ejemplo de una fábrica de filtros purificadores de agua. Se realiza el siguiente análisis:

\begin{itemize}
    \item Variables políticas:
    \begin{itemize}
        \item Un reciente cambio de presidente. Impacto positivo.
        \item Las elecciones de alcaldes y gobernadores a celebrarse en dos años. Sin impacto.
        \item Los acercamientos para concertar un tratado comercial con Centroamérica. Impacto muy positivo.
    \end{itemize}
    \item Variables económicas:
    \begin{itemize}
        \item Los cambios de divisa. Impacto muy negativo.
        \item El efecto que tiene el aumento en la tasa de interés. Impacto negativo.
        \item El aumenta en las exportaciones del último año. Impacto muy positivo.
    \end{itemize}
    \item Variables sociales:
    \begin{itemize}
        \item Cambio de pensamiento de la sociedad frente al autocuidado. Impacto positivo.
    \end{itemize}
    \item Variables tecnológicas:
    \begin{itemize}
        \item La impresión 3D como oportunidad para optimizar costes. Impacto positivo.
        \item Pérdida de información magnética. Impacto muy negativo.
    \end{itemize}
    \item Variables ambientales:
    \begin{itemize}
        \item El trabajo para obtener certificaciones ambientales. Sin impacto.
        \item El trabajo con los residuos de producción. Impacto negativo.
    \end{itemize}
    \item Variables legales:
    \begin{itemize}
        \item La implementación obligatoria del sistema de gestión en seguridad y salud en el trabajo. Sin impacto.
    \end{itemize}
\end{itemize}

\section{Define as ``Forzas de Porter'' e busca un exemplo onde se apliquen.}

El análisis de las cinco Fuerzas de Porter busca asesorar a una empresa sobre las fuerzas de competencia en su sector. Concretamente, de acuerdo con \cite{RefWorks:doc:602fb6588f0850dd1efa1bcd} analiza:

\begin{itemize}
    \item Rivalidad entre las empresas.
    \item Poder de negociación de los clientes.
    \item Poder de negociación de los proveedores.
    \item Amenaza de los nuevos competidores entrantes.
    \item Amenaza de productos sustitutos.
\end{itemize}

Por ejemplo, en \cite{RefWorks:doc:602fb9d78f087d63396a4d34} se hace un análisis de las cinco Fuerzas de Porter de Spotify:

\begin{itemize}
    \item Amenaza de nuevos competidores: Alta (amenaza). Hay muchos, y muy fuertes (Apple Music, Tidel, Deezer, Amazon Music...).
    \item Poder negociador de los clientes: Media (ni oportunidad ni amenaza). Los clientes tienen cierto poder porque hay mucha competencia, y es muy fácil cambiarse, pero por otro lado, el mercado está muy segmentado, y ninguna de sus rivales es una opción claramente mejor, lo cual resta poder de negociación a los consumidores.
    \item Amenaza de productos sustitutivos: Media (ni oportunidad ni amenaza). Como se indicaba, ninguna de sus rivales es una opción claramente mejor. El precio es más o menos igual, y el servicio también.
    \item Poder de negociación de los proveedores: Alta (amenaza). Los proveedores (músicos, discográficas) pueden decidir retirarle los derechos de difusión, como fue el caso de Taylor Swift, recientemente.
    \item Amenaza de productos sustitutos: Media-alta (amenaza). En base a todo lo anterior, se puede deducir que la amenaza es media-alta, ya que a dos fuerzas medias se suman otras dos fuerzas altas.
\end{itemize}

\section{Define a ``Análise DAFO'' e busca un exemplo.}

Según \cite{RefWorks:doc:602fb92d8f085533dc6835e1}, un análisis DAFO ``es una herramienta que permite al empresario analizar la realidad de su empresa, marca o producto para poder tomar decisiones de futuro''. Sus iniciales vienen de \textit{Debilidades, Amenazas, Fortalezas y Oportunidades}. Dice también que ``ayuda a establecer las estrategias [de un proyecto empresarial] para que éste sea viable''.

Se divide en dos partes: interno (Fortalezas y Debilidades), y externo (Amenazas y Oportunidades).

En \cite{RefWorks:doc:602fca5f8f085e432d8613ac}, se analiza el caso de Disney World:

    \begin{figure}[ht] 
        \label{DAFO} 
        \begin{minipage}[b]{0.5\linewidth}
          \centering

          \textbf{Debilidades.}
          \begin{itemize}
              \item Ticket de alto precio
              \item Nicho de mercado por edad
              \item Mercado geográfico limitado
              \item Altos costos operativos
          \end{itemize}

          \vspace{4ex}
        \end{minipage}%%
        \begin{minipage}[b]{0.5\linewidth}
          \centering
          
          \textbf{Amenazas.}
\begin{itemize}
    \item Competidores
    \item Accidentes (publicidad negativa)
    \item Cambio poblacional
\end{itemize}

          \vspace{4ex}
        \end{minipage} 
        \begin{minipage}[b]{0.5\linewidth}
          \centering
          
          \textbf{Fortalezas.}
\begin{itemize}
    \item Solidez financiera
    \item Líder de mercado
    \item Reconocimiento de marca
    \item Líder en animaciones (películas y films)
\end{itemize}

          \vspace{4ex}
        \end{minipage}%% 
        \begin{minipage}[b]{0.5\linewidth}
          \centering
          
          \textbf{Oportunidades.}
\begin{itemize}
    \item Lanzar nuevas historias, películas, personajes
    \item Nuevos avances tecnológicos (realidad virtual)
    \item Mercados emergentes en crecimiento
\end{itemize}

          \vspace{4ex}
    \end{minipage} 
\end{figure}








\bibliography{export}{}

\end{document}