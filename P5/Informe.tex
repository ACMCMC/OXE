\documentclass[a4paper]{article}

\usepackage[spanish]{babel}
\usepackage{listings}
\usepackage[utf8]{inputenc}
\usepackage{titling}
\usepackage{enumitem}
\usepackage{fancyhdr}
\usepackage{xcolor}
\usepackage{geometry}
\usepackage{graphicx}
\usepackage{hyperref}
\usepackage{cite}
\usepackage{url}
\usepackage{amsmath}

%%search -> (?:url=\{)(.*)(:?\})
%%replace -> howpublished={\\url{$1}}
\geometry{a4paper, margin=7em}


\lstset{
    frame=single,
    breaklines=true,
    numbers=left,
    keywordstyle=\color{blue},
    numbersep=15pt,
    numberstyle=,
    basicstyle=\linespread{1.5}\selectfont\ttfamily,
    commentstyle=\color{gray},
    stringstyle=\color{orange},
    identifierstyle=\color{green!40!black},
}

\setlength{\parindent}{4em}
%%\setlength{\parindent}{0em}
\setlength{\parskip}{0.8em}
    
%%\renewcommand{\familydefault}{phv} %%Seleccionamos Helvetica
    
\lstdefinestyle{console}
{
    numbers=left,
    backgroundcolor=\color{violet},
    %%belowcaptionskip=1\baselineskip,
    breaklines=true,
    %%xleftmargin=\parindent,
    %%showstringspaces=false,
    basicstyle=\footnotesize\ttfamily,
    %%keywordstyle=\bfseries\color{green!40!black},
    %%commentstyle=\itshape\color{green},
    %%identifierstyle=\color{blue},
    %%stringstyle=\color{orange},
    basicstyle=\scriptsize\color{white}\ttfamily,
}
    
\title{Interactiva 5}
\author{Aldán Creo Mariño}
    
    
\pagestyle{fancy}
\fancyfoot[R]{\thepage}
\fancyfoot[C]{}
\makeatletter
\let\runauthor\@author
\let\runtitle\@title
\makeatother
\fancyhead[L]{\runauthor}
\fancyhead[C]{OXE}
\fancyhead[R]{Interactiva 5}

\bibliographystyle{plain}
    
\begin{document}
\maketitle

\section{Escolle un sector dentro do CNAE, consulta previamente a clasificación CNAE e anota o seu código.}

He elegido el código 6201 (``Actividades de programación informática'').

\section{Calcula os rateos do cadro para o último ano dispoñible das dúas empresas.}

\begin{center}
    \begin{tabular}{ | p{5cm} | p{2cm} | p{2cm} | p{5cm} | }
        \hline
      & Bouge SA & Ingeciber SA & Fórmula rateo \\ \hline
      Fondo de rotación & 154606 & 78920 & $\text{Activo corrente} - \text{Pasivo corrente}$ \\ \hline
      Razón ou rateo de solvencia & 2.01 & 1.21 & $\frac{\text{Activo corrente}}{\text{Pasivo corrente}}$ \\ \hline
      Razón ou rateo de liquidez a curto prazo & 2.01 & 1.21 & $\frac{\text{Activo corrente} - \text{Existencias}}{\text{Pasivo corrente}}$ \\ \hline
      Razón ou rateo de tesourería & 1.5 & 0.4 & $\frac{\text{Tesourería} + \text{Inv. financeiros a c.p.}}{\text{Pasivo corrente}}$ \\ \hline
      Rendibilidade financeira & 55\% & 15\% & $\frac{\text{Resultado do exercicio}}{\text{Patrimonio neto}}*100$ \\ \hline
      Rendibilidade económica & 31\% & 1\% & $\frac{\text{Res. antes de xuros e impostos}}{\text{Activo total neto}}*100$ \\ \hline
      Rotación de activos totais & 0.31 & 0.004 & $\frac{\text{Ingresos de explotación}}{\text{Activo total}}$ \\ \hline
      Rotación de activo corrente & 0.37 & 0.017 & $\frac{\text{Ingresos de explotación}}{\text{Activo corrente}}$ \\ \hline
    \end{tabular}
\end{center}

\section{Analiza a información extraída dos rateos anteriores e elabora unha conclusión para ambas empresas da súa situación individual e comparativamente.}

En el caso de Bouge SA, el fondo de rotación es notablemente mayor que el de Ingeciber SA. Esto, en términos absolutos, nos indica que una cantidad mayor de activos en Bouge SA están financiados con recursos permanentes, es decir, el pasivo no corriente (como préstamos a largo plazo) y el patrimonio neto. Pese a todo, esta medida de por sí no nos aporta mucha información, ya que depende del volumen de la empresa.

Una medida mucho más informativa es el ratio de solvencia. En este caso, nos habla sobre la capacidad de una empresa de hacer frente a sus deudas, ya que mide la fracción que representa el fondo de rotación frente a los activos corrientes (tesorería, por ejemplo). Es decir, si la empresa tuviese que pagar todas sus deudas a corto plazo (pasivos a corto plazo) en ése momento, ¿podría hacerlo? ¿Y en qué medida?

\begin{itemize}
    \item Si el ratio de solvencia es mayor que 1, nos indica que el activo corriente es mayor que el pasivo corriente. Es decir, podría usar sus activos corrientes para pagar las deudas (al menos en teoría, ya que no todos son igual de líquidos). Si, por ejemplo, el ratio de solvencia es 2, éso quiere decir que la empresa podría pagar sus deudas y conservar la mitad de los activos corrientes que tenía.
    \item Si el ratio de solvencia fuese menor que 1, lo que significaría sería lo contrario: que la empresa no dispone del activo corriente necesario para pagar sus deudas a corto plazo en el momento. La única opción que tendría sería tratar de liquidar sus activos no corrientes, para poder pagar las deudas, pero este ratio no entra a evaluar eso (no tiene en cuenta los activos totales, que incluyen los activos no corrientes). La idea del ratio de solvencia, como comentaba, es hablar sobre la capacidad que tendría de pagar con sus activos corrientes (que por lo general son más o menos líquidos).
\end{itemize}

En el caso de las dos empresas, se puede observar que tienen un ratio de solvencia mayor que 1, pero en el caso de Bouge SA es mucho mejor que en el de Ingeciber SA. Como explicaba, esto nos indica que si Ingeciber SA y Bouge SA tuvieran que pagar sus deudas al cierre de las cuentas anuales, Bouge SA podría hacerlo conservando la mitad de sus activos corrientes, y así seguir adelante con un margen razonable, mientras que Ingeciber podría verse en problemas con mucha más facilidad, ya que pese a poder afrontar el pago, se quedaría casi sin activos corrientes, lo cual le causaría dificultades para afrontar nuevas deudas, o invertir para expandirse, por ejemplo. Una solución que podría tomar sería abrir una ronda de financiación, pero no es una solución segura.

En lo que se refiere al ratio de liquidez a corto plazo, es el mismo que el ratio de solvencia en el caso de las dos empresas, ya que no disponen de existencias. Esto es así porque se dedican a desarrollar programas informáticos, así que este ratio no es muy relevante en este caso. En un caso de otro tipo de empresa, sería un indicador más fiable de la capacidad de pago de sus deudas, ya que al restarle las existencias al activo corriente, nos quedamos con la parte más líquida del activo. Si yo soy una empresa, nadie me garantiza que consiga vender mis existencias, así que el ratio de solvencia no es una medida totalmente fiable de que pueda hacer frente a mis deudas.

Por ejemplo, podría ser el caso de una empresa que se cree ahora, aprovechando la situación del COVID, para fabricar mascarillas. Supongamos que compra fábricas con préstamos a corto plazo, para conseguir fondos rápidamente, y el proceso de contratación de personal, firma de contratos con proveedores y distribuidores, etc. le lleva unos meses. Si justo en julio el Gobierno dice que ya no hace falta llevar mascarillas, la demanda bajaría muchísimo, pese a que la empresa estaba lista para vender, imaginemos, un millón de mascarillas a un euro cada una. De repente, se vería con un montón de deudas que tiene que pagar a corto plazo, y nadie quiere comprar sus existencias. Aquí el ratio de solvencia podría ser muy positivo (imaginando que su pasivo corriente fuera considerablemente menor que el activo corriente). Podríamos pensar, entonces, que la empresa no tiene ningún tipo de problema, pero en realidad estaría al borde de la quiebra, en la medida en que no va a conseguir vender las mascarillas y pagar sus deudas. Aquí, el ratio que sería más revelador, sería el de liquidez a corto plazo, porque sería nuestra forma de poder ver que la grandísima parte de su activo corriente son las existencias que no va a poder vender, y fijándonos en ese ratio es como nos podríamos dar cuenta de que la empresa tiene un serio problema.

Un ratio más restrictivo aún es el de tesorería. En este caso, habla ``del dinero que tiene la empresa en sus manos''. Es decir, dentro Del grupo que conforman los activos corrientes, solamente tiene en cuenta Los más líquidos: es decir, las inversiones financieras a corto plazo que se pueden vender sin problemas, en general, y lo que se dispone en tesorería (bancos, caja...). Es el dinero del que dispone la empresa de forma casi segura. Por tanto, este ratio es incluso más fiable para evaluar la capacidad de pago de una empresa frente a sus deudas.

En el caso de nuestras dos empresas, podemos ver que Bouge SA tiene un ratio de 1.5, ya que en este caso no tenemos en cuenta sus derechos de cobro, que representarían ese 0.5 de diferencia. Los derechos de cobro son menos líquidos que tesorería y las inversiones financieras a corto plazo, ya que nuestros deudores podrían no pagarnos (por ejemplo, que nuestros clientes se declarasen en bancarrota). Como es lógico, por ese motivo se excluye del ratio de tesorería. En el caso de Ingeciber SA, este ratio es ciertamente preocupante, ya que es del 0.4. Esto implica que no sería capaz de hacer frente a sus deudas con el dinero del que dispone directamente. La gran diferencia entre el ratio de liquidez a corto plazo y el ratio de tesorería se debe a que a Ingeciber SA le deben 307471 euros, que entrarían en el primero, pero se excluyen del segundo. Es decir, necesitaría conseguir ejecutar sus derechos de cobro para poder pagar sus deudas, y esto es menos seguro que disponer del dinero a través de sus cuentas bancarias, por ejemplo. Una opción si no se viese capaz de pagar sus deudas sería intentar renegociarlas.

La rentabilidad financiera representa el beneficio neto que se obtiene con respecto al patrimonio neto. Es una medida que se usa para valorar la ganancia que obtienen los inversores con relación a lo que invierten. Si se usan fondos ajenos, es decir, si la empresa se endeuda para poder invertir, entonces baja el beneficio neto, ya que hay que pagar los intereses. En cambio, si la inversión la hace con sus fondos propios (patrimonio neto), entonces el beneficio neto es superior. Es decir, como indicaba, esta medida es útil para poder evaluar la rentabilidad hacia los inversores, porque se fija en el dinero que queda al final de todo, en comparación con el patrimonio neto de la empresa (principalmente, sus fondos propios). Si obtiene un resultado positivo con respecto a los fondos propios de los que dispone, esto repercutirá en un mayor retorno para los inversores.

Como se puede ver en la tabla, la rentabilidad financiera es mucho superior en Bouge SA que en Ingeciber SA. Conviene comparar esta medida con la rentabilidad económica para sacar mejores conclusiones.

La rentabilidad económica lo que representa es el beneficio obtenido con respecto a los activos de la empresa. Es decir, si una empresa tiene pocos activos, pero obtiene un beneficio muy grande, su rentabilidad económica será muy alta, y viceversa. Por ejemplo, una empresa que venda caviar, y lo cobre al triple de lo que le cuesta producirlo, y apenas tenga activos (poco dinero en caja, una fábrica pequeña etc.), tendrá una rentabilidad económica elevadísima: está consiguiendo muchos beneficios sin tener apenas activos.

En el caso de las dos empresas, la que tiene mayor rentabilidad económica es Bouge SA (31\%), mientras que Ingeciber SA tiene solo un 1\% de rentabilidad económica. ¿Qué nos indica esto, en comparación también con la rentabilidad financiera?

En el caso de Bouge SA, lo que podemos observar es que la rentabilidad financiera es mayor que la económica, por lo que podemos deducir que la empresa se está endeudando para poder realizar inversiones, pero que pese a todo el coste de las deudas no impide que los inversores obtengan un beneficio. Esto se puede entender si nos fijamos en que en la fórmula de la rentabilidad financiera tiene en cuenta el resultado del ejercicio, por lo que se descontarían intereses de préstamos, por ejemplo, y el patrimonio neto (entre el cual está el capital social). En cambio, la fórmula de la rentabilidad económica tiene en cuenta el resultado antes de juros e impuestos, por lo que aquí no entrarían los intereses por préstamos, por seguir con el ejemplo anterior, y después se divide por el activo total neto, que se vería engrosado por el activo que hemos conseguido con financiación externa (es decir, la rentabilidad económica baja). Por tanto, si estamos endeudándonos para poder financiar nuestras inversiones, puede darse el caso de que tengamos rentabilidad económica o no, pero es importante tener en cuenta también si esa rentabilidad económica se traduce en una rentabilidad financiera (de cara a los inversores). El hecho de que la rentabilidad financiera sea superior a la económica nos habla de que el coste medio de las deudas que adquirimos es menor que el beneficio que tenemos por habernos endeudado de esa forma. Esto se conoce como apalancamiento positivo.

En el caso de Ingeciber SA, ambas cifras son mucho menores. Podemos deducir que también se está endeudando para poder obtener beneficios, pero sus beneficios de cara a los inversores, y también por la explotación de sus actividades, son comparativamente inferiores a los de Bouge SA. Sumado a lo que hemos visto antes de que tiene una menor liquidez, podemos hacernos a la idea de que esta empresa está en una peor situación, con respecto a Bouge SA.

Finalmente, la rotación de activos nos habla acerca de la capacidad de la empresa de usar sus activos para generar ventas. Por ejemplo, si mi empresa tiene 1€ de activos, y genera 365€ de ventas (simplificando), quiere decir que en un año, ``rota'' sus activos una vez al día (se generan ventas por el valor de los activos una vez al día).

Más concretamente, tenemos dos medidas: la rotación de activos totales, y la rotación de activo corriente. La diferencia entre ellas está en el nombre: una tiene en cuenta todos los activos, y la otra sólo los corrientes.

En el caso de Bouge SA, tenemos una rotación de activos totales de 0.31, y de activo corriente de 0.37. Es decir, unas tres veces al año, más o menos cada cuatro meses, se generan unas ventas por valor del total de los activos de la empresa, y esto se da cada tres meses (aproximadamente) con respecto a los activos corrientes. Por tanto, los activos están sirviendo para generar unas ventas relativamente considerables. Por tanto, los activos están sirviendo para generar unas ventas relativamente considerables. Siempre podrían ser mayores, pero también podrían ser más escuetas. La rotación del activo corriente es mayor que la de activos totales, pero no de forma muy considerable. Esto lo que nos indica es que la mayoría de ventas que se realizan se están apoyando en activos medianamente líquidos que tenga la empresa, ya que la rotación que toma en cuenta en los activos totales se fija también en activos no corrientes (vehículos para el transporte, por ejemplo) que tenga la empresa. Es decir, la rotación que se fija en los activos corrientes pone el foco en los activos con los que realmente podemos invertir (dinero que tengamos guardado en el banco, por ejemplo).

Para el caso de Ingeciber SA, se aplica el mismo razonamiento expuesto, solo que teniendo en cuenta que la periodicidad es mucho menor, especialmente cuando tenemos en cuenta todos los activos. Es decir, está siendo mucho menos eficiente a la hora de generar ventas con sus activos.

Si comparamos las rotaciones de los activos con las rentabilidades de las empresas, podemos ver que en el caso de Bouge SA, tenemos valores superiores en ambos casos. Esto lo que nos indica es que está siendo más eficiente a la hora de generar rentabilidad, ya que rota más sus activos (genera más ventas). También podría haber sido el caso de que tuviera una rotación más baja, lo que implicaría que por cada venta tendría que obtener una rentabilidad mucho mayor.

El caso de Ingeciber SA es peor que el de Bouge SA, porque rota menos sus activos (vende menos), y probablemente por ello obtiene una menor rentabilidad. Podría ser que tuviera una rentabilidad relativamente elevada por cada venta, pero si son muy pocas, entonces su rentabilidad seguirá siendo escasa, y por las cifras que vemos, tampoco parece que sea el caso. Lo que sabemos es que vende menos y que los beneficios que consigue no son suficientes para superar a la rentabilidad de Bouge SA.

En mi opinión, Bouge SA debería concentrarse en mejorar su rotación de activos (vender más), para poder aumentar así su rentabilidad. En lo referente al ratio de solvencia y demás indicadores asociados, no parecen muy preocupantes. En principio, no debería verse en problemas para afrontar sus deudas.

En cambio, Ingeciber SA debería tratar por todos los medios de conseguir cobrar a sus deudores, para mejorar su ratio de tesorería (podría serle difícil pagar sus deudas), y de esta forma poder invertir más en sus actividades, mejorando el número de ventas y también su rentabilidad, y liberándose en gran medida de la dependencia en activos ajenos que tiene actualmente.

En general, habiendo analizado los parámetros indicados, la conclusión es bastante clara: si tuviéramos que invertir en una de las dos empresas, la opción más segura sería Bouge SA, ya que todos sus indicadores son más positivos que los de Ingeciber SA. En la primera podrían siempre mejorarse los Ratios, como en cualquier empresa empresa, pero no considero que sean especialmente preocupantes. En la primera podrían siempre mejorarse los indicadores, como en cualquier empresa, pero no considero que sean especialmente preocupantes. En la segunda, sí que hay algunos (especialmente el ratio de tesorería, y la rotación de los activos) que son motivo de preocupación, y podría ser una inversión bastante arriesgada.


\end{document}